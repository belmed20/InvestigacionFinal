%% BioMed_Central_Tex_Template_v1.06
%%                                      %
%  bmc_article.tex            ver: 1.06 %
%                                       %

%%IMPORTANT: do not delete the first line of this template
%%It must be present to enable the BMC Submission system to
%%recognise this template!!

%%%%%%%%%%%%%%%%%%%%%%%%%%%%%%%%%%%%%%%%%
%%                                     %%
%%  LaTeX template for BioMed Central  %%
%%     journal article submissions     %%
%%                                     %%
%%          <8 June 2012>              %%
%%                                     %%
%%                                     %%
%%%%%%%%%%%%%%%%%%%%%%%%%%%%%%%%%%%%%%%%%


%%%%%%%%%%%%%%%%%%%%%%%%%%%%%%%%%%%%%%%%%%%%%%%%%%%%%%%%%%%%%%%%%%%%%
%%                                                                 %%
%% For instructions on how to fill out this Tex template           %%
%% document please refer to Readme.html and the instructions for   %%
%% authors page on the biomed central website                      %%
%% http://www.biomedcentral.com/info/authors/                      %%
%%                                                                 %%
%% Please do not use \input{...} to include other tex files.       %%
%% Submit your LaTeX manuscript as one .tex document.              %%
%%                                                                 %%
%% All additional figures and files should be attached             %%
%% separately and not embedded in the \TeX\ document itself.       %%
%%                                                                 %%
%% BioMed Central currently use the MikTex distribution of         %%
%% TeX for Windows) of TeX and LaTeX.  This is available from      %%
%% http://www.miktex.org                                           %%
%%                                                                 %%
%%%%%%%%%%%%%%%%%%%%%%%%%%%%%%%%%%%%%%%%%%%%%%%%%%%%%%%%%%%%%%%%%%%%%

%%% additional documentclass options:
%  [doublespacing]
%  [linenumbers]   - put the line numbers on margins

%%% loading packages, author definitions

%\documentclass[twocolumn]{bmcart}% uncomment this for twocolumn layout and comment line below
\documentclass{bmcart}

%%% Load packages
%\usepackage{amsthm,amsmath}
%\RequirePackage{natbib}
%\RequirePackage[authoryear]{natbib}% uncomment this for author-year bibliography
%\RequirePackage{hyperref}
\usepackage[utf8]{inputenc} %unicode support
%\usepackage[applemac]{inputenc} %applemac support if unicode package fails
%\usepackage[latin1]{inputenc} %UNIX support if unicode package fails


%%%%%%%%%%%%%%%%%%%%%%%%%%%%%%%%%%%%%%%%%%%%%%%%%
%%                                             %%
%%  If you wish to display your graphics for   %%
%%  your own use using includegraphic or       %%
%%  includegraphics, then comment out the      %%
%%  following two lines of code.               %%
%%  NB: These line *must* be included when     %%
%%  submitting to BMC.                         %%
%%  All figure files must be submitted as      %%
%%  separate graphics through the BMC          %%
%%  submission process, not included in the    %%
%%  submitted article.                         %%
%%                                             %%
%%%%%%%%%%%%%%%%%%%%%%%%%%%%%%%%%%%%%%%%%%%%%%%%%


\def\includegraphic{}
\def\includegraphics{}



%%% Put your definitions there:
\startlocaldefs
\endlocaldefs


%%% Begin ...
\begin{document}

%%% Start of article front matter
\begin{frontmatter}

\begin{fmbox}
\dochead{Investigación}

%%%%%%%%%%%%%%%%%%%%%%%%%%%%%%%%%%%%%%%%%%%%%%
%%                                          %%
%% Enter the title of your article here     %%
%%                                          %%
%%%%%%%%%%%%%%%%%%%%%%%%%%%%%%%%%%%%%%%%%%%%%%

\title{El efecto liberador de la música}

%%%%%%%%%%%%%%%%%%%%%%%%%%%%%%%%%%%%%%%%%%%%%%
%%                                          %%
%% Enter the authors here                   %%
%%                                          %%
%% Specify information, if available,       %%
%% in the form:                             %%
%%   <key>={<id1>,<id2>}                    %%
%%   <key>=                                 %%
%% Comment or delete the keys which are     %%
%% not used. Repeat \author command as much %%
%% as required.                             %%
%%                                          %%
%%%%%%%%%%%%%%%%%%%%%%%%%%%%%%%%%%%%%%%%%%%%%%

\author[
   addressref={aff1},                   % id's of addresses, e.g. {aff1,aff2}
   corref={aff1},                       % id of corresponding address, if any
   noteref={n1},                        % id's of article notes, if any
   email={hector-1996@hotmail.es}   % email address
]{\inits{JE}\fnm{Beltran Medrano} \snm{Hector Alonso}}


%%%%%%%%%%%%%%%%%%%%%%%%%%%%%%%%%%%%%%%%%%%%%%
%%                                          %%
%% Enter the authors' addresses here        %%
%%                                          %%
%% Repeat \address commands as much as      %%
%% required.                                %%
%%                                          %%
%%%%%%%%%%%%%%%%%%%%%%%%%%%%%%%%%%%%%%%%%%%%%%

\address[id=aff1]{%                           % unique id
  \orgname{Tijuana B.C.}, % university, etc
  \street{ITT},                     %
  %\postcode{}                                % post or zip code
  \city{Baja California},                              % city
  \cny{Mexico}                                    % country
}


%%%%%%%%%%%%%%%%%%%%%%%%%%%%%%%%%%%%%%%%%%%%%%
%%                                          %%
%% Enter short notes here                   %%
%%                                          %%
%% Short notes will be after addresses      %%
%% on first page.                           %%
%%                                          %%
%%%%%%%%%%%%%%%%%%%%%%%%%%%%%%%%%%%%%%%%%%%%%%

\begin{artnotes}
%\note{Sample of title note}     % note to the article
\note[id=n1]{Equal contributor} % note, connected to author
\end{artnotes}

\end{fmbox}% comment this for two column layout

%%%%%%%%%%%%%%%%%%%%%%%%%%%%%%%%%%%%%%%%%%%%%%
%%                                          %%
%% The Abstract begins here                 %%
%%                                          %%
%% Please refer to the Instructions for     %%
%% authors on http://www.biomedcentral.com  %%
%% and include the section headings         %%
%% accordingly for your article type.       %%
%%                                          %%
%%%%%%%%%%%%%%%%%%%%%%%%%%%%%%%%%%%%%%%%%%%%%%

\begin{abstractbox}

\begin{abstract} % abstract
\parttitle{Resumen} %if any
\linebreak A lo largo de su vida una persona adopta un estilo de vida unico en el cual lo caracteriza como persona unica y por ello no existe otra igua, pero las personas han ido desarrollando un estilo de bienestar con la musica que escuchan y por ello se sienten edentificados con ellas mismas.

En esta investigacion se busca informar a los lectores como es que la musica puede influir en nuestras vidas asi como conocer como es que hemos desarrollado ese interes tan apasionado por algunos estilos de musican.

\end{abstract}

%%%%%%%%%%%%%%%%%%%%%%%%%%%%%%%%%%%%%%%%%%%%%%
%%                                          %%
%% The keywords begin here                  %%
%%                                          %%
%% Put each keyword in separate \kwd{}.     %%
%%                                          %%
%%%%%%%%%%%%%%%%%%%%%%%%%%%%%%%%%%%%%%%%%%%%%%

\begin{keyword}
\kwd{Musica}
\kwd{Emociones}
\kwd{audicion}
\end{keyword}

% MSC classifications codes, if any
%\begin{keyword}[class=AMS]
%\kwd[Primary ]{}
%\kwd{}
%\kwd[; secondary ]{}
%\end{keyword}

\end{abstractbox}
%
%\end{fmbox}% uncomment this for twcolumn layout

\end{frontmatter}

%%%%%%%%%%%%%%%%%%%%%%%%%%%%%%%%%%%%%%%%%%%%%%
%%                                          %%
%% The Main Body begins here                %%
%%                                          %%
%% Please refer to the instructions for     %%
%% authors on:                              %%
%% http://www.biomedcentral.com/info/authors%%
%% and include the section headings         %%
%% accordingly for your article type.       %%
%%                                          %%
%% See the Results and Discussion section   %%
%% for details on how to create sub-sections%%
%%                                          %%
%% use \cite{...} to cite references        %%
%%  \cite{koon} and                         %%
%%  \cite{oreg,khar,zvai,xjon,schn,pond}    %%
%%  \nocite{smith,marg,hunn,advi,koha,mouse}%%
%%                                          %%
%%%%%%%%%%%%%%%%%%%%%%%%%%%%%%%%%%%%%%%%%%%%%%

%%%%%%%%%%%%%%%%%%%%%%%%% start of article main body
% <put your article body there>

%%%%%%%%%%%%%%%%
%% Background %%
%%

\section*{Introducción}

Investigacion que nos explica la influencia de la música en el aprendizaje humano.
Desde los comienzos de la humanidad, se han hecho evidentes los efectos que la música ha tenido sobre la conducta humana.
Podemos diferenciar la música, desde el punto de vista en que afecta a las personas, en dos tipos principales, musica relajante, que se caracteriza por tener un ritmo regular, y provocar un efecto sedante; y musica  estimulante que estimula las emociones. 
Nadie sabe por qué la música tiene un efecto tan potente en nuestras emociones, pero gracias a estudios recientes ya contamos con algunas claves.

Nos gusta la música porque nos hace sentir bien. ¿Y esto por qué sucede?, La influencia de cada uno de los elementos de la música esta presente e  la mayoría de los individuos y en esta investigación hablaremos de algunos de ellos.
%\cite{koon,oreg,khar,zvai,xjon,schn,pond,smith,marg,hunn,advi,koha,mouse}

\section*{Justificacion}
Investigar las causas y consecuencias de la musica en nuestro estado de ánimo, asi como poder demostrar alguna relación con nuestra personalidad y la forma en que la adoptamos en nuestros estilos de vida.
\subsection*{Objetivo general}
Para demostrar que la musica tiene un efecto en la vida de las personas.
\subsection*{Objetivos especificos}
-Demostrar que la musica influye en el comportamiento de la musica.
\linebreak -Informar a los lectores que hay medios para sentirse mejor escuchando musica.
\section*{Hipotesis}
La musica puede influenciar en el comportamiento de las personas, asi como en nuestros estados de animo.
\section*{Marco Teórico}
\section*{El efecto liberador de la música en las emociones}
Nos gusta la música porque nos hace sentir bien. ¿Y esto por qué sucede?
En 2001, los neurocientíficos Anne Blood y Robert Zatorre de la Universidad McGill en Montreal demostraron que las personas que escuchan música placentera activa regiones del cerebro llamadas límbicas y paralímbicas, que están conectadas a respuestas eufóricas, como las que experimentamos con el sexo, la buena comida o las drogas adictivas.
Estas respuestas proceden del impulso generado por el neurotransmisor conocido como dopamina.
\subsection*{¿Supervivencia?}
Es fácil entender por qué el sexo y la comida generan esta respuesta de la dopamina que nos hace querer más, contribuyendo a nuestra supervivencia y propagación.
¿Pero por qué sucede también con los sonidos, que nada tienen que ver con nuestra supervivencia?
La verdad nadie la sabe. Sin embargo, ahora tenemos más claves de por qué la música produce intensas emociones.
La teoría favorita actualmente entre los científicos que estudian cómo nuestra mente procesa la música se remonta a 1956, cuando el filósofo y compositor Leonard Meyer sugirió que la emoción en la música tiene que ver con lo que esperamos y con si lo obtenemos o no.
\subsection*{Expectativas}
Meyer elaboró unas teorías psicológicas de la emoción que proponían que éstas surgen cuando somos incapaces de satisfacer un deseo.
Esto, como pueden imaginar, genera enfado y frustración, pero cuando encontramos lo que queremos, ya sea amor o un cigarrillo, la recompensa es dulce.
Según Meyer esto es lo que hace la música. Establece patrones de sonido y regularidades que tratan de provocarnos predicciones inconscientes sobre lo que se viene.
Si hemos acertado el cerebro obtiene un premio, que es el flujo de dopamina.
El baile entre expectativas envuelve el cerebro con un manto de emociones placenteras.
¿Pero por qué nos deberían importar estas expectativas? No es que nuestra vida dependa de ello.
\subsection*{Emociones indefinidas}
La teoría de Meyer se vio reforzado por el estudio llevado a cabo por Zatorre y sus colegas, que demostró que la respuesta estimulada por la música depende de la comunicación entre "emoción" y "lógica" en el cerebro.
Pero nuestra respuesta a la música debe estar también condicionada por tantos otros factores: si la escuchamos solos o en grupo, o si asociamos una canción a una experiencia determinada.
Otras veces ni siquiera podemos reconocer qué emociones nos genera la música.
Podemos reconocer una canción triste sin sentirnos tristes. Incluso si nos sentimos tristes no es como una tristeza arrebatadora, y la música puede disfrutarse aun provocando lágrimas.
Algunas piezas de Bach pueden generar emociones intensas incluso cuando no podemos determinar qué clase de emociones son.
Así que nunca terminaremos de entender por qué la música estimula nuestras emociones, al menos hasta que no sepamos mejor cómo es nuestro mundo emocional.
\section*{El efecto de la música sobre el ser humano}
La influencia de cada uno de los elementos de la música en la mayoría de los individuos es la siguiente:
\subsection*{Tiempo}
los tiempos lentos, entre 60 y 80 pulsos por minuto, suscitan impresiones de calma, de sentimentalismo, serenidad, ternura y tristeza. Los tiempos rápidos de 100 a 150 pulsos por minuto, suscitan impresiones alegres, excitantes y vigorosas.
\subsection*{Ritmo}
los ritmos lentos inducen a la paz y a la serenidad, y los rápidos suelen producir la activación motora y la necesidad de exteriorizar sentimientos, aunque también pueden provocar situaciones de estrés.
\subsection*{Armonía}
Se da al sonar varios sonidos a la vez. A todo el conjunto se le llama acorde. Los acordes consonantes están asociados al equilibrio, el reposo y la alegría. Los acordes disonantes se asocian a la inquietud, el deseo, la preocupación y la agitación.
\subsection*{Tonalidad}
los modos mayores suelen ser alegres, vivos y graciosos, provocando la extroversión de los individuos. Los modos menores presentan unas connotaciones diferentes en su expresión e influencia. Evocan el intimismo, la melancolía y el sentimentalismo, favoreciendo la introversión del individuo.
\subsection*{La altura}
las notas agudas actúan frecuentemente sobre el sistema nervioso provocando una actitud de alerta y aumento de los reflejos. También ayudan a despertarnos o sacarnos de un estado de cansancio. El oído es sensible a las notas muy agudas, de forma que si son muy intensas y prolongadas pueden dañarlo e incluso provocar el descontrol del sistema nervioso. Los sonidos graves suelen producir efectos sombríos, una visión pesimista o una tranquilidad extrema.
\subsection*{La intensidad}
es uno de los elementos de la música que influyen en el comportamiento. Así, un sonido o música tranquilizante puede irritar si el volumen es mayor que lo que la persona puede soportar.
\subsection*{La instrumentación}
los instrumentos de cuerda suelen evocar el sentimiento por su sonoridad expresiva y penetrante. Mientras los instrumentos de viento destacan por su poder alegre y vivo, dando a las composiciones un carácter brillante, solemne, majestuoso. Los instrumentos de percusión se caracterizan por su poder rítmico, liberador y que incita a la acción y el movimiento.
\section*{Música y emociones}
Desde la antigüedad la música se ha considerado como un arte. Es un código, un lenguaje universal, que está presente en todas las culturas de la historia de la humanidad. Curiosamente, los signos jeroglíficos que representaban la palabra “música” eran idénticos a aquellos que representaban los estados de “alegría” y “bienestar”. Y en China, los dos ideogramas que la representan, significan “disfrutar del sonido”. Por lo tanto, hay una gran coincidencia en relación a los significados sobre lo que es la música, que ha perdurado a lo largo del tiempo, donde predominan las sensaciones agradables y placenteras que produce.
\subsection*{La Musicoterapia}
Posiblemente, los orígenes de la utilización terapéutica de los sonidos y la música se remonten al principio de la humanidad. Ya Platón citaba que “la música era para el alma lo que la gimnasia para el cuerpo”, reconociendo que ésta poseía determinadas cualidades o propiedades que incidían en nuestras dimensiones emocional y/o espiritual.


La American Music Therapy Association (AMTA) define la musicoterapia como “una profesión, en el campo de la salud, que utiliza la música y actividades musicales para tratar las necesidades físicas, psicológicas y sociales de personas de todas las edades. La Musicoterapiamejora la calidad de vida de las personas sanas y cubre las necesidades de niños y adultos con discapacidades y enfermedades. Sus intervenciones pueden diseñarse para mejorar el bienestar, controlar el estrés, disminuir el dolor, expresar sentimientos, potenciar la memoria, mejorar la comunicación y facilitar la rehabilitación física”

Así, si consideramos a la enfermedad como una ruptura, desequilibrio o una falta de comunicación, podemos pensar que la música puede ayudar a tender los puentes necesarios para que esa comunicación que se encuentra bloqueada, fluya; contribuyendo al restablecimiento o mejora de la salud.

En la actualidad, la musicoterapia se aplica a un amplio campo en relación a diversos trastornos, dirigida a personas de todas las edades. Son frecuentes las aplicaciones en la educación (Autismos, hiperactividad, síndrome de Down…), salud mental (depresión, ansiedad, estrés…), medicina (oncología, dolor, personas en la UCI…) y geriatría (demencias…)

Gracias a la capacidad de la música de actuar a todos los niveles, con la musicoterapia se pueden conseguir algunos objetivos como:

-Mejora el nivel de afectividad y conducta.

-Desarrollar la comunicación y medios de expresión.

-Liberar energía reprimida.

-Desarrollar la sensibilización afectiva-emocional.

-Dotar a las personas de vivencias musicales enriquecedoras que ayuden a motivarse.

-Reforzar la autoestima y personalidad.

-Rehabilitar, socializar y educar.

\subsection*{¿Afecta la música a nivel emocional?}
¿Quién no ha experimentado en alguna ocasión cierta emoción mientras escuchaba música? El sonido y la música nos producen emociones, y éstas, modifican nuestra fisiología, nuestras hormonas, alteran nuestro ritmo cardíaco y pulsaciones. Existen multitud de momentos en los que utilizamos la música, ya sea de forma consciente e inconsciente.

La música se empleaba en la antigüedad para animar a guerreros y cazadores. Incluso, en el cine se utiliza como medio para multiplicar los efectos de determinadas escenas, convirtiéndose en un código indispensable para la caracterización emocional del guión y las situaciones (Cohen, 2011).

Nuestro estado de ánimo, muchas veces se ve reflejado por el tipo de música que escuchamos o entonamos. Una canción triste puede inducirnos a un estado melancólico, mientras que una canción alegre puede excitarnos y proporcionarnos unos minutos de felicidad. Al igual que una música suave y armónica nos acompaña en nuestros momentos de relajación y estudio o una música rítmica nos estimula mientras hacemos ejercicio.

También la música tiene efecto en muchos de nuestros recuerdos importantes. ¿Quién no ha asociado alguna vez determinada situación con un tema musical?

Las áreas cerebrales que se activan con las emociones y la música son prácticamente las mismas. Cuando el cerebro percibe las ondas sonoras, se producen ciertas reacciones psicofisiológicas. Así, respondemos con emociones y éstas provocan alteraciones fisiológicas como el aumento de la segregación de neurotransmisores y otras hormonas, que actúan sobre el sistema nervioso central.

La música puede modificar nuestros ritmos fisiológicos, alterar nuestro estado emocional y ser capaz de cambiar nuestra actitud mental, aportando paz y armonía a nuestro espíritu. La música ejerce una poderosa influencia sobre el ser humano a todos los niveles.
\section*{¿Qué le hace la música a nuestro cerebro?}
La música, que en principio es sustancia física, influye en muchos   aspectos biológicos y de comportamiento del ser humano. Quizá la influencia más llamativa sea la que ejerce en nuestro cerebro, que es plástico y susceptible de adaptación: el estudio y práctica de la música puede modificarlo para conseguir que sus dos hemisferios funcionen con más agilidad e integración, de modo más holístico. No sólo en funciones musicales, sino también en dominios como la memoria o la matemática.

Para que la sociedad española pueda beneficiarse de ello, es  necesaria la educación en esta disciplina desde temprana edad, y el lugar idóneo para que llegue a todos es la enseñanza obligatoria.

Por otro lado, el consumo generalizado de música en nuestra sociedad, mal utilizado, produce daños en la salud, incluso lesiones irreversibles. Educar para preservar la salud respecto de un medio hoy omnipresente, utilizado de modo constante como reclamo para el consumo, es otra de las grandes tareas que esta materia debe abordar.

Este estudio trata de una serie de efectos de la música en la formación integral del individuo, que son quizás difíciles de asumir para quien no haya sido educado musicalmente, pero que vamos a exponer a la luz de recientes investigaciones que aportan datos objetivos para ilustrar esta afirmación, la cual engloba factores ya conocidos como el desarrollo del ritmo, de la coordinación psicomotriz, de  sensibilidad artística… junto a otras aportaciones que sobre las que se trabaja en la actualidad mediante diversos trabajos de investigación.
\paragraph*{ }
“El cerebro de los músicos es distinto: El cerebro es un órgano plástico que se moldea con relación a los estímulos culturales que recibe desde la infancia. … en el cerebro de los músicos, la zona cuya función es registrar y diferenciar los estímulos acústicos es un 25 más grande que en el de las personas que jamás hayan tocado un instrumento.”

\subsection*{¿CAMBIA NUESTRO CEREBRO CON LA EDUCACIÓN MUSICAL?}
Pues sí, esta pregunta ha podido responderse afirmativamente mediante la utilización en investigaciones recientes sobre Neurofisiología las modernas técnicas de la Magneto Encefalograma (MEG), la Tomografía por Emisión de Positrones (PET) o la
Imagen de Resonancia Magnética Funcional (FMRI), con las que se pueden visualizar las partes del cerebro implicadas en las distintas tareas que realiza este órgano, y que han permitido llegar a las siguientes conclusiones:

Frances Rauscher sugiere que la música estimula conexiones neuronales específicas situadas en el centro de razonamiento abstracto del cerebro, lo que hace a los individuos más inteligentes.
Según el Dr. Schlaug, el cerebelo (zona del cerebro que contiene el 70 por ciento de las neuronas) es un 50 por ciento más grande en los músicos que en otros grupos.
Según un estudio de la Universidad de Hong Kong3 los adultos que han recibido enseñanza musical antes de los 12 años tienen mejor memoria oral porque tienen más desarrollado el lóbulo temporal izquierdo del cerebro.
\section*{Efectos de la música en el cerebro}
Desde siempre se han atribuido poderes casi mágicos a la música y hay hasta quienes aseguran que calma a las fieras, lo que a veces se tomaba como broma, pero gracias a investigaciones que se realizaron sobre el tema hoy es posible tener la certeza de que su influencia va más allá y que en verdad incide en la estabilidad emocional y la salud física.
Investigadores han constatado que la vibración de los sonidos musicales inciden directamente en la corteza cerebral y llegan incluso a apoyar el tratamiento de enfermedades del corazón, llevan a la persona a evocar colores, sabores y olores, además de que eleva el IQ verbal y permite una mejor comunicación.
Cuando escuchamos música, la vibración se procesa en diferentes áreas del cerebro. A principios de los años noventa empezaron a realizarse estudios para conocer el grado de participación que tiene el cerebro en ello. Sin embargo, gracias a un mapeo detallado, hoy es posible afirmar que las distintas notas musicales inciden de manera diferente en cada una de las áreas que componen nuestra “computadora cerebral”.
Por ejemplo, la corteza sensorial, que localizan en la parte externa y superior del cerebro tiene una relación directa con la danza y la ejecución de instrumentos musicales; mientras que la llamada corteza visual, en la parte central, posibilita la lectura de notas y la apreciación de movimientos que efectúan quienes hacen música.
Un estudio realizado en la Universidad de Londres estableció que la música afecta la percepción que tenemos de las cosas, así que quien escucha música alegre tiende a ver las situaciones cotidianas de manera positiva, alegre o entusiasta, mientras que quienes optan por canciones o temas tristes llenan su vida de melancolía y ésta la proyectan hacia la imagen que tiene de las demás personas.
En la Universidad de California, investigadores pidieron a grupos de personas nacidas en Estados Unidos y México que escucharan algunos temas musicales y los resultados obtenidos fueron muy similares: quienes estuvieron en contacto con notas alegres asociaron la experiencia a colores vibrantes, que denotaban alegría y vida; en tanto a quienes se les puso música triste, percibieron colores oscuros, sobrios, neutros. 
Los resultados fueron los mismos sin importar a qué grupo pertenecían las personas que apoyaban el estudio.
\section*{Conclusion}
Emoción, expresión, habilidades sociales, teoría de la mente, habilidades lingüísticas y matemáticas, habilidades visoespaciales y motoras, atención, memoria, funciones ejecutivas, toma de decisiones, autonomía, creatividad, flexibilidad emocional y cognitiva, todo confluye en forma simultánea en la experiencia musical compartida. Las personas cantan y bailan juntas en todas las culturas. Sabemos que lo hacemos hoy y lo seguiremos haciendo en el futuro.

Es cierto que la musica puede influenciar en nuestas emociones ya que es algo que tenemos desde el momento en el que somos bebes y nos cantan canciones de cuna.
Determinado estudios demuestran que la musica influye en el comportamiento de las personas 
\section*{Trabajo futuro}
Los posibles trabajos que se puedan generar en un futura hacerca de este tema seria mas enfocado en la musicoterapia o las formas en que la musica pueda afectar nuestra forma de ser.

Para muchas personas que han escuchado musica toda su vida se podrian hacer investigaciones si algun tipo de musica ayude en general al mejoramiento de nuestro sistema auditivo ademas de relajarnos o mantenernos de buen humor.

%%%%%%%%%%%%%%%%%%%%%%%%%%%%%%%%%%%%%%%%%%%%%%
%%                                          %%
%% Backmatter begins here                   %%
%%                                          %%
%%%%%%%%%%%%%%%%%%%%%%%%%%%%%%%%%%%%%%%%%%%%%%

\begin{backmatter}


%%%%%%%%%%%%%%%%%%%%%%%%%%%%%%%%%%%%%%%%%%%%%%%%%%%%%%%%%%%%%
%%                  The Bibliography                       %%
%%                                                         %%
%%  Bmc_mathpys.bst  will be used to                       %%
%%  create a .BBL file for submission.                     %%
%%  After submission of the .TEX file,                     %%
%%  you will be prompted to submit your .BBL file.         %%
%%                                                         %%
%%                                                         %%
%%  Note that the displayed Bibliography will not          %%
%%  necessarily be rendered by Latex exactly as specified  %%
%%  in the online Instructions for Authors.                %%
%%                                                         %%
%%%%%%%%%%%%%%%%%%%%%%%%%%%%%%%%%%%%%%%%%%%%%%%%%%%%%%%%%%%%%

% if your bibliography is in bibtex format, use those commands:
\bibliographystyle{bmc-mathphys} % Style BST file (bmc-mathphys, vancouver, spbasic).
% Bibliography file (usually '*.bib' )
% for author-year bibliography (bmc-mathphys or spbasic)
% a) write to bib file (bmc-mathphys only)
% @settings{label, options="nameyear"}
% b) uncomment next line
%\nocite{label}

% or include bibliography directly:
% \begin{thebibliography}
% \bibitem{b1}
% \end{thebibliography}

%%%%%%%%%%%%%%%%%%%%%%%%%%%%%%%%%%%
%%                               %%
%% Figures                       %%
%%                               %%
%% NB: this is for captions and  %%
%% Titles. All graphics must be  %%
%% submitted separately and NOT  %%
%% included in the Tex document  %%
%%                               %%
%%%%%%%%%%%%%%%%%%%%%%%%%%%%%%%%%%%

%%
%% Do not use \listoffigures as most will included as separate files

\newpage
\renewcommand{\refname}{Referencias}\begin{thebibliography}{X}
	\bibitem{Baz} \textsc{http://www.bbc.com/mundo/noticias/2014/04/140429\_salud\_musica\_placer\_aa}
	\bibitem{Baz} \textsc{http://santiagocarlosiad.wordpress.com/2011/12/29/el-efecto-de-la-musica-sobre-el-ser-humano/}
	\bibitem{Baz} \textsc{https://lamenteesmaravillosa.com/musica-y-emociones/}
	\bibitem{Baz} \textsc{http://elpais.com/elpais/2015/08/31/ciencia/1441020979\_017115.html}
	\bibitem{Baz} \textsc{http://educrea.cl/cerebro-ymusica/}
	\bibitem{Baz} \textsc{http://www.telemundochicago.com/noticias/salud/efectos-de-la-musica-en-el-cerebro-se-traducen-en-mayor-bienestar-emocional-286365971.html}
	\bibitem{Baz} \textsc{http://diasgrandiosos.com/es\_US/ama/cosas-que-la-musica-puede-revelarte-sobre-la-personalidad-de-tus-hijos.html}
	\bibitem{Dan} \textsc{http://infobae.com/2015/08/09/1746939-que-revela-el-gusto-musical-la-personalidad}
	\bibitem{Dan} \textsc{http://archivo.de10.com.mx/wdetalle3005.html}
	\bibitem{Dan} \textsc{http://es.slideshare.net/xchelmusica/la-influencia-de-la-msica-en-la-personalidad-12753029}
\end{thebibliography}


%%%%%%%%%%%%%%%%%%%%%%%%%%%%%%%%%%%
%%                               %%
%% Tables                        %%
%%                               %%
%%%%%%%%%%%%%%%%%%%%%%%%%%%%%%%%%%%

%% Use of \listoftables is discouraged.
%%


%%%%%%%%%%%%%%%%%%%%%%%%%%%%%%%%%%%
%%                               %%
%% Additional Files              %%
%%                               %%
%%%%%%%%%%%%%%%%%%%%%%%%%%%%%%%%%%%



\end{backmatter}
\end{document}
